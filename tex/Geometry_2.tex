\documentclass{report}
\usepackage{../style}
 
\begin{document}
\section{The Affine Plane}
The $n$-dimensional \emph{affine space} $\AA^n$ is the space of $n$-tuples of complex numbers.\footnote{I'm not sure why the notational difference from $\CC^n$ is necessary.
In Fulton, $\AA^n(k) = k^n$ for any field $k$}.
We call $\AA^2$ the affine plane.

If $f(x_1,x_2)$ is an irreducible complex polynomial in two variables, the \emph{locus of zeros} of $f$ is called a \emph{plane curve}, or an \emph{affine plane curve}.
Using vector notation $x = \prn{x_1,x_2}$, 
\[ X = \cbr{x \mid f(x) = 0}. \]
The degree of curve $X$ is the degree of its irreducible defining polynomial.

Note, the word curve here is derived from the \emph{algebraic} dimension of one: the only proper subsets of a plane curve that can be defined by polynomial equations are finite sets.
We will use words that sort of generalize geometric intuition from the reals to affine space over other fields.

We will inspect the projection of a plane curve $X$ by inspecting its projection to the affine line.
To do this, write its defining polynomial $f$ as a polynomial in $x_2$ whose coefficients are polynomials in $x_1$:
\[
  f(x_1,x_2) = c_0(x_1)x_2^d + c_1(x_1)x_2^{d-1} + \dots + c_d(x_1).
\]
Suppose that $f$ isn't a polynomial in $x_1$ alone, so that $d$ is positive.
The \emph{fibre} of a map $X \rightarrow Z$ over a point $q$ of $Z$ is defined to be the preimage of $q$.
The fibre of a projection $X \rightarrow \AA^1$ over a point $x_1 = a$ is the set of points $\prn{a,b}$ such that $b$ is a root of the one-variable polynomial
\[
  f(a,x_2) = c_0(a)x_2^d + \dots + c_d(a).
\]
There will be finitely many points in the fibre, and the fibre won't be empty unless $f(a,x_2)$ is a constant.
Hence $X$ covers most of the $x_1$-line finitely often.

In order to classify plane curves, we allow linear changes of variable and translations.
Hence, we can change variables by the following, where $Q$ is an invertible $2 \times 2$ complex matrix and $a$ is a complex translation vector:
\[ Qx' + a = x. \]
We change polynomial equations $f(x) = 0$ to $f(Qx' + a) = 0$.
We also allow scalar multiplication of polynomials.
Using this, all \emph{lines} (plane curves of degree 1) become equivalent.

An \emph{affine conic} is a plane curve of degree two.
Every equation $q(x_1,x_2) = 0$ in which $q$ is an irreducible quadratic polynomial is equivalent, by change of coordinates, to one of the following equations:
\[ x_1^2 - x_2^2 - 1 = 0 \text{   or   } x_1^2 - x_2 = 0. \]

The proof of this is similar to the one used in classifying real conics, and one might call these hyperbola and parabola.
The complex ellipse $x_1^2 + x_2^2 - 1 = 0$ becomes a hyperbola when one multiplies $x_2$ by $i$.

A surprising fact: there are infinitely many inequivalent cubic curves.
Cubic polynomials in two variables depend on the coefficients of ten monomials in $x_1,x_2$, and linear operators, translations, and scalar multiplication only give seven parameters to work with, leaving three essential parameters.

\section{The Projective Plane}

\end{document}

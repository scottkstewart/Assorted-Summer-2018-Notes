\documentclass{report}
\usepackage{../style}
 
\begin{document}
\section{The Affine Plane}
The $n$-dimensional \emph{affine space} $\AA^n$ is the space of $n$-tuples of complex numbers.\footnote{I'm not sure why the notational difference from $\CC^n$ is necessary.
In Fulton, $\AA^n(k) = k^n$ for any field $k$}.
We call $\AA^2$ the affine plane.

If $f(x_1,x_2)$ is an irreducible complex polynomial in two variables, the \emph{locus of zeros} of $f$ is called a \emph{plane curve}, or an \emph{affine plane curve}.
Using vector notation $x = \prn{x_1,x_2}$, 
\[ X = \cbr{x \mid f(x) = 0}. \]
The degree of curve $X$ is the degree of its irreducible defining polynomial.

Note, the word curve here is derived from the \emph{algebraic} dimension of one: the only proper subsets of a plane curve that can be defined by polynomial equations are finite sets.
We will use words that sort of generalize geometric intuition from the reals to affine space over other fields.

We will inspect the projection of a plane curve $X$ by inspecting its projection to the affine line.
To do this, write its defining polynomial $f$ as a polynomial in $x_2$ whose coefficients are polynomials in $x_1$:
\[
  f(x_1,x_2) = c_0(x_1)x_2^d + c_1(x_1)x_2^{d-1} + \dots + c_d(x_1).
\]
Suppose that $f$ isn't a polynomial in $x_1$ alone, so that $d$ is positive.
The \emph{fibre} of a map $X \rightarrow Z$ over a point $q$ of $Z$ is defined to be the preimage of $q$.
The fibre of a projection $X \rightarrow \AA^1$ over a point $x_1 = a$ is the set of points $\prn{a,b}$ such that $b$ is a root of the one-variable polynomial
\[
  f(a,x_2) = c_0(a)x_2^d + \dots + c_d(a).
\]
There will be finitely many points in the fibre, and the fibre won't be empty unless $f(a,x_2)$ is a constant.
Hence $X$ covers most of the $x_1$-line finitely often.

In order to classify plane curves, we allow linear changes of variable and translations.
Hence, we can change variables by the following, where $Q$ is an invertible $2 \times 2$ complex matrix and $a$ is a complex translation vector:
\[ Qx' + a = x. \]
We change polynomial equations $f(x) = 0$ to $f(Qx' + a) = 0$.
We also allow scalar multiplication of polynomials.
Using this, all \emph{lines} (plane curves of degree 1) become equivalent.

An \emph{affine conic} is a plane curve of degree two.
Every equation $q(x_1,x_2) = 0$ in which $q$ is an irreducible quadratic polynomial is equivalent, by change of coordinates, to one of the following equations:
\[ x_1^2 - x_2^2 - 1 = 0 \text{   or   } x_1^2 - x_2 = 0. \]

The proof of this is similar to the one used in classifying real conics, and one might call these hyperbola and parabola.
The complex ellipse $x_1^2 + x_2^2 - 1 = 0$ becomes a hyperbola when one multiplies $x_2$ by $i$.

A surprising fact: there are infinitely many inequivalent cubic curves.
Cubic polynomials in two variables depend on the coefficients of ten monomials in $x_1,x_2$, and linear operators, translations, and scalar multiplication only give seven parameters to work with, leaving three essential parameters.

\newpage
\section{The Projective Plane}
The $n$-dimensional \emph{projective space} $\PP^n$ is the set of equivalence classes of \emph{nonzero} vectors $x = \prn{x_0,\dots,x_n}$, with the equivalence relation
\[
  \prn{x_0',\dots,x_n'} \sim \prn{x_0,\dots,x_n} \text{\hspace{5pt} if \hspace{5pt}}
  \prn{x_0',\dots,x_n'} = \prn{\lambda x_0,\dots, \lambda x_n}
\]
for some nonzero complex number $\lambda$. 
The equivalence classes are points of $\PP^n$ and one often refers to a point by a particular vector in its classes.
There is an obvious bijective correspondence between points of $\PP^n$ and one-dimensional subspaces of $\CC^{n+1}$.
We will use the words \emph{line} and \emph{plane} as before, referring to one and two-dimensional projective space.

Points of the projective line $\PP^1$ are equivalence classes of nonzero vectors $(x_0,x_1)$.
If $x_0$ isn't zero, we normalize the first component, otherwise we normalize the second.
Hence the points are written uniquely as $(1,u)$ for $u \in \CC$ and $(0,1)$;
the projective line $\PP^1$ can be obtained by adding the last point, a \emph{point at infinity}, to the affine \emph{u}-line $\UU$, which is a complex plane.
Topologically, the projective line is a 2-sphere.

A \emph{line} $L$ is projective space $\PP^n$ can be described in terms of a pair of distinct points $p$ and $q$, as the set of points $\cbr{rp + sq}$, with $r,s \in \CC^*$.
The points of this line correspond bijectively to the points of $\PP^1$ by
\[
  rp + sq \longleftrightarrow \prn{r,s}.
\]
We can also describe a line in the projective plane as the locus of solutions to a homogeneous linear equation
\[
  s_0x_0 + s_1x_1 + s_2x_2 = 0.
\]

\begin{lemma}
  Two distinct lines in the projective plane have exactly one point in common, and a pair of distinct points is contained in exactly one line.\footnote{The proof for this was omitted.
  This is clear through the bijective correspondence with subspaces of complex space, as well as intuition from Euclidean geometry.}\qed
\end{lemma}

We can work similarly with $\PP^2$ as we did with $\PP^1$;
if $x_0$ is nonzero, we can write 
\[\prn{x_0,x_1,x_2} \sim \prn{1,u_1,_2} \longleftrightarrow (u_1,u_2).\]
This gives the affine plane as a subset of $\PP^2$, and we denote this subset by $\UU^0$.
We say the points of $\UU^0$ (those with $x_0$ nonzero) are the \emph{points at finite distance}, where the \emph{points at infinity} of $\PP^2$ are on the \emph{line at infinity} $L^0$, the locus $\cbr{x_0 = 0}$.

There is an analogous correspondence between the points with $x_1$ or $x_2$ nonzero and an affine plane $\AA^2$ for each.
We denote the subset $\cbr{x_i \neq 0}$ by $\UU^i$.
We call the three sets $\UU^0$, $\UU^1$, $\UU^2$ the \emph{standard covering} of $\PP^2$ by three \emph{standard affine open sets}.
Note: which points of $\PP^2$ depend on which of the standard affine open sets is at finite distance.
With $(x_0,x_1,x_2)$, Artin may define $\cbr{x_0 \neq 0}$ at finite distance, but for $(x,y,z)$, he may define $\cbr{z \neq 0}$ at finite distance.

The \emph{real projective plane} $\RR \PP^2$ is the set of equivalence classes of nonzero real vectors $(x_0,x_1,x_2)$, the equivalence relation being $(x') \sim (x)$ if $(x') = \lambda(x)$ for some nonzero real number $\lambda$.
It can be thought of as the space of one-dimensional subspaces of a real vector space of dimension three.
Let $V$ denote $\RR^3$ and let $U$ be the plane $\cbr{x_0 = 1} \subset V$
Then $U$ is analogous to the subset $U^0$ of the complex projective plane $\PP^2$.

We can project $V$ from the origin to $U$, sending a point $(x_0,x_1,x_2)$ of $V$ to the point $(1,u_1,u_2)$ with $u_i = x_i/x_0$; then the fibres of this projection are the one-dimensional subspaces of $\RR^3$, the origin omitted.\footnote{This is the ``picture plane'' from art.}
This projection to $U$ is undefined at the points $(0,x_1,x_2)$; lines through the origin that are orthogonal to the $x_0$ axis don't meet $U$, and they correspond to the points at infinity of $\RR\PP^2$.

An invertible $3 \times 3$ matrix $P$ determines a linear change of coordinates in $\PP^2$;
\[ Px' = x. \]
As the next proposition shows, four special points, $e_0 = (1,0,0)^t$, $e_1 = (0,1,0)^t$, $e_2 = (0,0,1)^t$, and $e = (1,1,1)$ determine the coordinates.

\begin{proposition}
  Let $p_0$, $p_1$, $p_2$, $q$ be four points of $\PP^2$, no three of which lie on a line.
  There is, up to scalar factor, a unique linear coordinate change $Px' = x$ such that $Pp_i = e+i$ and $Pq = e$.
\end{proposition}
\begin{proof}
  Since $p_0,p_1,p_2$ don't lie on a line, those three vectors are independent, and they span $\CC^3$.
  Then, $q$ is a combination $c_0p_0 + c_1p_1 + c_2p_2$, and because no three points lie on a line, $c_i$ are nonzero.
  Then we can scale the vectors (by nature of $\PP^2$) so that $q = p_0 + p_1 + p_2$; then making the $i$th column of $P$ be $p_i$ gives $Pe_i = p_i$ and $Pe = q$; uniqueness up to scaling is clear by inspection.
\end{proof}

A polynomial $f(x_0,x_1,x_2)$ is \emph{homogeneous of degree $d$} if all monomials that appear with nonzero coefficient have degree $d$.
A \emph{conic} is the locus of zeros of an irreducible homogeneous quadratic polynomial of three variables, a combination of the six monomials
\[
  x_0^2,x_1^2,x_2^2,x_0x_1,x_1x_2,x_00x_2.
\]
\begin{proposition}
  For any conic $C$, there is a choice of coordinates so that $C$ becomes the locus
  \[ x_0x_1 + x_0x_2 + x_1x_2 = 0. \]
\end{proposition}
\begin{proof}
  Since $C$ isn't a line, it will contain three non-colinear points;
  change coordinates so that these are $e_0$, $e_1$, and $e_2$.
  Let $f$ be the quadratic in those coordinates with zero locus $C$.
  Then $f(1,0,0) = 0$, so the coefficient of $x_0^2$ is zero; similar logic gives zero coefficient for all squared terms.
  $f$ has the form
  \[ f = ax_0x_a + bx_0x_2 + cx_1x_2, \]
  and we may scale the variables to make $a=b=c=1$.
\end{proof}
 
\newpage
\section{Plane Projective Curve}


\end{document}

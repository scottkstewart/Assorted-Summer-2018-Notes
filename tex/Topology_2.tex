\documentclass[10pt]{report}
\usepackage{../style}

\begin{document}
\section{Topological Spaces and Bases (\S M12-13)}
\subsection{Topological Spaces}
\begin{definition}
  A \emph{topology} on a set $X$ is a collection $\ST$ of subsets of $X$ having the following properties:
  \begin{enumerate}[label={(\arabic*)}]
    \item $\varnothing$ and $X$ are in $\ST$.
    \item The union of any subcollection of $\ST$ is in $\ST$.
    \item The intersection of any finite subcollection of $\ST$ is in $\ST$.
  \end{enumerate}
  A set $X$ for which a topology $\ST$ has been specified is a \emph{topological space}, and any subset $U \subset X$ is open if $U \in \ST$.
\end{definition}

  Examples of this include the \emph{discrete topology}, where every subset is open, and the \emph{indiscrete} or \emph{trivial topology}, where only $\varnothing$ and $X$ are open.
  
  Another important (probably) example is the \emph{finite complement topology}:
  this is $(X,\ST_f)$, where $\ST_f$ is the collection of subsets $U \subset X$ such that $X - U$ is either finite or all of $X$.
  It's easy to check that $X$, $\emptyset$ work, so we note the following facts for an indexed family of nonempty elements $\{U_\alpha\} \subset \ST_f$:
  \[ X - \bigcup U_\alpha = \bigcap (X - U\alpha;) \]
  \[ X - \bigcap_{i=1}^n U_i = \bigcup_{i = 1}^n (X - U_i). \]
  From this, one can see that arbitrary unions and finite intersections work. 

\begin{definition}
  Suppose that $\ST$, $\ST'$ are topologies on $X$.
  If $\ST' \supset \ST$, then $\ST'$ is \emph{finer} than $\ST$ (and \emph{strictly finer} with a proper inclusion).
  The reverse inclusion implies $\ST'$ is \emph{coarser} than $\ST$.
  If either of these inclusions is true, then $\ST'$ is \emph{comparable} to $\ST$.
\end{definition}

Sometimes fully specifying the elements of a topology is difficult or annoying--like in algebra, one usually specifies a topology by a subcollection of open sets which generate the whole thing.

\subsection{Bases}
\begin{definition}
  If $X$ is a set, a \emph{basis} for a topology on $X$ is a collection $\SB$ of subsets of $X$ (called \emph{basis elements}) such that
  \begin{enumerate}[label={(\arabic*)}]
    \item For each $x \in X$, there exists some $B \in \SB$ with $x \in B$.
    \item If $x \in B_1 \cap B_2$, then there exists $B_3 \in \SB$ with $x \in B_3 \subset B_1 \cap B_2$.
  \end{enumerate}
  
  If $\SB$ satisfies these conditions, then we define the \emph{topology $\ST$ generated by $\SB$} by the following:
  $U \subset X$ is open if, for each $x \in U$, there is some $B \in \SB$ with $x \in B \subset U$.
\end{definition}

  We will show that this is a topology.
  Note that $\varnothing$ satisfies the condition for openness vacuously, and all $x \in X$ are contained in a basis element by hypothesis, so $X$ is open as well.
  All that's left is that unions and finite intersections work.

  Suppose $\{U_\alpha\} \subset \ST$ is a collection of open sets.
  Then, for any element $x \in \bigcap U_\alpha$, $x \in U_\alpha$ for some $\alpha$; choose $x \in \SB$ with $x \in B \subset U_\alpha$ and note that this is also in the union.
  For the intersection, we will prove the intersection of two and use induction; suppose $U,V \in \ST$.
  Then, for any $x \in U \cap V$, choose $B_1,B_2 \in \ST$ containing $x$ so that $B_1 \subset U$ and $B_2 \subset V$; then, by definition, there exists a $B_3 \subset B_1 \cap B_2$ with $x \in B_3 \subset \ST$, giving openness.
  Induction is straightforward.

  \begin{lemma}
    Let $X$ be a set and let $\SB$ be a basis for a topology $\ST$ on $X$.
    Then $\ST$ is the collection of all unions of elements of $\SB$.
  \end{lemma}
  \begin{proof}
    That $\ST$ includes all unions is straightforward from the fact that $\ST$ is a topology.
    To show that any $U \in \ST$ is a union of elements, choose for each $x \in U$ some $B_x \in \SB$ such that $x \in B_x \subset U$.
    Then $U = \bigcup B_x$, so $U$ is a union of elements of $\SB$.
  \end{proof}

  There is one significant, potentially counter-intuitive fact to note about bases:
  the representation of any open subset as a union of basis elements is not unique.
  Topological bases seem similar to the algebraic notion of a spanning set or generating set.

  \begin{lemma}
    Let $X$ be a topological space.
    Suppose that $\SC$ is a collection of open sets of $X$ such that, for each $X$ of $X$ and each $x \in U$, there is an element $C \in \SC$ such that $x \in C \subset U$.
    Then $\SC$ is a basis for the topology of $X$.
  \end{lemma}
  \begin{proof}
    First we check that $\SC$ is a basis.
    The first condition is easy; since $X$ is open, there is an element $C \in \SC$ with $x \in C \subset X$.
    To show the intersection property, take any $x \in C_1 \cap C_2$, where $C_i \in \SC$.
    Then, since $C_1 \cap C_2$ is open, there exists an element $C_3 \in \SC$ with $x \in C_3 \subset C_1 \cap C_2$.

    Suppose $\ST$ is the set of open sets on $X$ and $\ST'$ is generated by $\SC$.
    Note that, for any $U \subset \ST$ with $x \in U$, there exists an element $C \in \SC$ with $x \in C \subset U$, so $\ST'$ is finer than $\ST$.
    Conversely, if $W \in \ST'$, then $W$ is a union of elements of $\SC$, which are each open, so $W$ is open and $\ST$ is finer than $\ST'$, giving $\ST = \ST'$.
  \end{proof}

  \begin{lemma}
    Let $\SB$ and $\SB'$ be bases for the topologies $\ST$ and $\ST'$ on $X$.
    Then the following are equivalent:
    \begin{enumerate}[label={(\arabic*)}]
      \item $\ST'$ is finer than $\ST$
      \item For each $x \in X$ and each basis element $B \in \SB$ containing $x$, there is a basis element $B' \in \SB'$ such that $x \in B' \subset B$.
    \end{enumerate}
  \end{lemma}
  \begin{proof}
    $(2) \implies (1)$. Suppose $U \in \ST$.
    Then for each $x \in U$, there is a $B_x \in \ST$ containing $x$ as a subset of $U$.
    By hypothesis, there exists a $B_x' \in \ST'$ with $x \in B_x' \subset B_x \subset U$, so $U$ is the union of elements of $\SB'$ and $U \in \ST'$.

    $(1) \implies (2)$. Suppose $U \in \ST$.
    Then, for each $x \in U$ and $B \in \SB$ containing $x$, by condition $(1)$ and the definition of generation, $B \in \ST'$; since $\ST'$ is generated by $\SB'$, there is an element $B' \in \SB'$ such that $x \in B' \subset B$.
  \end{proof}
  
  We will move on to our first serious examples of topological spaces:
  \begin{definition}
    If we denote the collection of all open intervals on the real line $\SB$, 
    \[ \prn{a,b} = \cbr{x \mid a < x < b }, \]
    the topology generated by $\SB$ is the \emph{standard topology} on the real line.
    This is the topology assumed for $\RR$ unless otherwise specified.

    If $\SB'$ is the half open intervals, we will call the topology generated by $\SB'$ the \emph{lower limit topology} on $\RR$; denote $\RR$ with this topology by $\RR_l$.

    If we let $K$ be the set of numbers of the form $1/n$ and let $\SB''$ be the collection of all open intervals $(a,b)$ and sets of the form $(a,b) - K$, we call the topology generated by $\SB''$ the \emph{K-topology} on $\RR$, denoted by $\RR_k$.
  \end{definition}

  \begin{lemma}
    The topologies of $\RR_l$ and $\RR_k$ are strictly finer than the standard topology on $\RR$, but are not comparable with one another.
  \end{lemma}
  \begin{proof}
    Let $\ST$, $\ST_l$, and $\ST_k$ be the topologies on $\RR$, $\RR_l$, and $\RR_k$.
    To show that $\ST_l$ is finer than $\ST$, take any $(a,b) \in \ST$ and $x \in (a,b)$; then $[x,b)$ contains $x$ and lies in $(a,b)$.
    However, no $(c,d) \subset [x,b)$ contains $x$, so $\ST_l \supsetneq \ST$.
    To show that $\ST_k$ is finer than $\ST$, any $(a,b) \in \ST$ has $(a,b) \in \ST_k$ by construction.
    However, $(-1,1) - K$ has no subset in $\ST$ containing 0, giving $\ST_k \supsetneq \ST$.
  
    To see that these are not comparable, use a similar argumen to before.
    There is no $(c,d) - K \subset [x,b)$ containing $x$, and there is no $[a,b) \subset (-1,1)-k$ contining $0$.
  \end{proof}

  One last basic\footnote{This was intended to be a really bad pun} definition:
  \begin{definition}
    A subbasis $\CS$ for a topology on $X$ is a collection of subsets of $X$ whose union equals $X$.
    The \emph{topology generated by the subbasis $\CS$} is defined to be the collection $\ST$ of all unions of finite intersections of the elements of $\CS$.
  \end{definition}

  We'll check that $\ST$ is a topology.
  By an earlier lemma, it suffices to show that the collection of finite intersections of $\CS$ is a basis.
  Given $x \in X$, it belongs to an element of $\CS$ and hence an element of $\SB$, giving the first condition.
  To show the second condition, let $B_1 = S_1 \cap \dots \cap S_m$ and $B_2 = S_1' \cap \dots \cap S_n'$ be any two elements of $\SB$; then $B_1 \cap B_2 = S_1 \cap \dots \cap S_m \cap S_1' \cap \dots \cap S_n' \in \SB$, making $\SB$ a basis.
 

\section{Closed Sets and Limit Points}
I'll defer talking about particular examples until later.
Refer to ``Basic Examples of Topological Spaces'' at the end of the chapter for definitions of order, product, subspace topologies, etc.
First I'll cover closed sets, limit points, and the Hausdorff axiom.

\subsection{Closed Sets}
We say a subset of a topological space is \emph{closed} if its complement is open.
\begin{theorem}
  Let $X$ be a topological space.
  Then the following conditions hold:
  \begin{enumerate}[label={(\arabic*)}]
    \item $\varnothing$ and $X$ are closed.
    \item Arbitrary intersections of closed sets are closed.
    \item Finite unions of closed sets are closed.\footnote{Use DeMorgan's Law}\qed
  \end{enumerate}
\end{theorem}

Note: while this is rarely done, a collection of closed sets satisfying these theorems can be used to specify a topology rather than open sets.

\begin{theorem}
  Let $Y$ be a subspace of $X$.
  Then a set $A$ is closed in $Y$ iff it is the intersection of a closed set of $X$ with $Y$.
\end{theorem}
\begin{proof}
  Suppose that $A = Y \cap C$ for some $C$ which is closed in $X$.
  Then there exists some $U = X - C$ which is open in $X$, and for which
  \[
    U \cap Y = Y - C \cap Y = Y - A
  \]
  is open in $Y$, meaning $A$ is closed in $Y$.
  
  Conversely, suppose $A$ is closed in $Y$.
  Then $Y - A$ is open, so there exists an open $U$ such that
  \[
    U \cap Y = Y - A
  \]
  and $X - U$ is closed in $X$, so that $A = (X - U) \cap Y$, the intersection of $Y$ and a closed set of $X$.\footnote{Ahh this proof is messy as hell, but the idea is there}
\end{proof}

\begin{theorem}
  Let $Y$ be a subspace of $X$.
  If $A$ is closed in $Y$ and $Y$ is closed in $X$, then $A$ is closed in $X$.\footnote{The proof was ``left to the reader,'' but this is an obvious finite intersection on closed sets in $X$ using the previous theorem.}\qed
\end{theorem}

\subsection{Closure and Interior of a Set}
We'll talk about the closure and interior, functionally the largest open subset and smallest closed subset.
\begin{definition}
  Given a subset $A$ of a topological space $X$, the \emph{interior} of $A$ is the union of all open sets contained in $A$, and the \emph{closure} of $A$ is the intersection of all closed sets containing $A$.
\end{definition}

Note the following inclusion, with $\Int A = A$ if open and $A = \bar A$ if closed:
\[
  \Int A \subset A \subset \bar A.
\]
In the situation where $A$ is a subset of $Y$ which is a subspace of $X$, we'll use care, and generally define the closure $\bar A$ as the closure of $A$ in $Y$.
More on this follows:

\begin{theorem}
  Let $Y$ be a subspace of $X$,
  let $A$ be a subset of $Y$,
  and let $\bar A$ denote the closure of $A$ in $X$.
  Then the closure of $A$ in $Y$ equals $\bar A \cap Y$.
\end{theorem}
\begin{proof}
  Denote the closure of $A$ in $Y$ as $B$.
  Since $\bar{A}$ is closed, $\bar{A} \cap Y$ is closed and contains $A$;
  then, $B \subset \bar A \cap Y$ by property of the intersection.

  On the other hand, since $B$ is closed in $Y$, $B = C \cap Y$ for some $C$ closed in $X$;
  $C$ is a closed set of $C$ containing $A$;
  then $\bar{A} \subset C$, so $(\bar A \cap Y) \subset (C \cap Y) = B$
\end{proof}

We'll write that a set $A$ \emph{intersects} a set $B$ if the intersection $A \cap B$ is nonempty.

\begin{theorem}
  Let $A$ be a subset of the topological space $X$.
  \begin{enumerate}[label={(\alph*)}]
    \item Then $x \in \bar A$ iff every open set $U$ containing $x$ intersects $A$.
    \item Supposing the topology of $X$ is given by a basis, then $x \in \bar A$ iff every basis element $B$ containing $x$ intersects $A$.
  \end{enumerate}
\end{theorem}
\begin{proof}
  We'll prove (a) using contraposition;
  if $x \notin \bar{A}$, then $U = X-\bar{A}$ is an open set in $X$ which contains $x$ and doesn't intersect $A$.
  If $U$ contains $x$ and doesn't intersect $\bar{A}$, then $X - U$ is a closed set containing $A$; then $X - U$ must contain $\bar A$, so $x \notin \bar A$.

  The rest follows from this.
  If not every basis element containing $x$ intersects $A$, then by (a), $x \notin \bar A$.
  If every basis element containing $x$ intersects $\bar A$, then so does every open set containing $x$ by definition of bases (each open set containing $x$ contains a basis element containing $x$).
\end{proof}

We'll write ``$U$ is an open set containing $x$'' as ``$U$ is a \emph{neighborhood} of $x$.''
Using this, we can rewrite the previous theorem as ``if $A$ is a subset of a topological space $X$, then $x \in \bar{A}$ iff every neighborhood of $X$ intersects $A$.''

\subsection{Limit Points}
We seek another way of describing closures.
\begin{definition}
  If $A$ is a subset of a topological space $X$, and $x$ is a point of $x$, we call $x$ a \emph{limit point} (or cluster point, point of accumulation) if every neighborhood of $x$ intersects $A$ in some point other than $x$ itself.
  In other words, $x$ is a limit point of $A$ if it belongs to the closure of $A - \cbr{x}$.
\end{definition}

\begin{theorem}
  Let $A$ be a subset of the topological space $X$ and let $A'$ be the set of all limit points of $A$. Then
  \[
    \bar A = A \cup A'.
  \]
\end{theorem}
\begin{proof}
  If $x \in A'$, every neighborhood of $x$ intersects $A$, so $A' \subset \bar A$.
  Furthermore, by definition, $A \subset \bar A$, so $A \cup A' \subset \bar A$.

  Let $x \in \bar A$; if $x \in A$ then $x \in A \cup A'$ trivially.
  Otherwise, every neighborhood of $x$ intersects $A$ at a point other than $x$, meaning $x \in A'$ so $x \in A \cup A'$ as desired.
\end{proof}
\begin{corollary}
  A subset of a topological space is closed iff it contains its limit points.\qed
\end{corollary}

\subsection{Hausdorff Spaces}
Note: not all topological spaces have closed one-point sets, and not all convergent sequences in general topological spaces converge to a unique point.
Examples of this come from the topology on $\cbr{a,b,c}$ indicated in figure 17.3 of Munkres.
We get rid of these with an extra axiom:
\begin{definition}
  A topological space $X$ is classed a \emph{Hausdorff space} if, for each pair $x_1,x_2$ of distinct points of $X$, there exist neighborhoods $U_1$ and $U_2$ of $x_1$ and $x_2$ which are disjoint.
\end{definition}

\begin{theorem}
  Every finite point set in a Hausdorff space is closed.
\end{theorem}
\begin{proof}
  It's sufficient to show that one-point sets are closed.
  Consider the one point set $\cbr{x_1}$ and any other element $x_2 \in X$.
  By definition, there is a neighborhood $U_2$ of $x_2$ which is disjoint from a neighborhood of $x_1$, which must contain $x_1$, i.e. $x_1 \notin U_2$; then $x_2$ isn't a limit point of $\cbr{x_1}$, so it is closed.
  Finite unions of closed one-point sets give the theorem.
\end{proof}

In fact, that finite-point sets are closed is weaker than the Hausdorff condition; we call this the \emph{$T_1$ axiom}.
In particular, $\RR$ with the finite complement topology is not a Hausdorff space, but it satisfies the $T_1$ axiom by construction.
We won't use the $T_1$ axiom very much, but it's important now:

\begin{theorem}
  Let $X$ be a space satisfying the $T_1$ axiom and let $A \subset X$.
  Then $x$ is a limit point of $A$ iff every neighborhood of $x$ contains infinitely many points of $A$.
\end{theorem}
\begin{proof}
  If every neighborhood of $x$ intersects $A$ at infinitely many points then $x$ is clearly a limit point of $A$.

  Suppose that $x \in X$ and suppose some neighborhood $U$ of $x$ intersects $A$ in finitely many points.
  Let $U \cap (A - \{x\}) = \cbr{x_1,\dots,x_n}$;
  then $X - \cbr{x_1 \dots x_n}$ is an open set of $X$ and
  \[ U \cap (X - \cbr{x_1,\dots,x_n})\]
  is a neighborhood of $x$ which doesn't intersect $A$.
  Hence $x$ is not a limit point of $A$ and the theorem follows.
\end{proof}

\begin{theorem}
  Let $x_n$ be a sequence of points in a Hausdorff space $X$ which converges; then $x_n$ converges to one point.
\end{theorem}
\begin{proof}
  Suppose $x_n \rightarrow x$ and $y \neq x$.
  Let $U$ and $V$ be disjoint neighborhoods of $x$ and $y$; then all but finitely many points of $x_n$ fall within $U$, so the same can't be true of $V$, and $x_n$ doesn't converge to $y$.
\end{proof}

\begin{theorem}
  The following examples are true:
  \begin{enumerate}[label={(\alph*)}]
    \item Every simply ordered set is a Hausdorff space in the order topology.
    \item The product of two Hausdorff spaces is a Hausdorff space.
    \item A subspace of a Hausdorff space is a Hausdorff space.
 \end{enumerate}
\end{theorem}
\begin{proof}
  Suppose $X$ is simply ordered with the order topology and $x,y \in X$.
  Without loss of generality, assume $x < y$; if there is an intermediate element $x < c < y$ then choose $(-\infty,c)$ and $(c,\infty)$ for disjoint neighborhoods; if no such element exists, then $(-\infty,y)$ and $(x,\infty)$ are disjoint open neighborhoods of $x$ and $y$ and the order topology is Hausdorff.

  Suppose $X$ and $Y$ are Hausdorff spaces and consider the product topology on $X \times Y$.
  Take any $x \times y , x' \times y' \in X \times Y$ distinct and neighborhoods $U,V,U',V'$ around those points with $U,U'$ if $x \neq x'$ and $V,V'$ disjoint if $y \neq y'$ (at least one of these is true).
  Then, $U \times V$ and $U' \times V'$ are disjoint neighborhoods of the points and the product topology is Hausdorff.

  Suppose $Y$ is a subspace of a Hausdorff space $X$, and suppose $x,y \in A$.
  Let $U,V$ be disjoint neighborhoods of $x,y$ in $X$;
  then $U \cap Y$ and $V \cap Y$ are disjoint neighborhoods of $x,y$ in $Y$ and the subspace topology is Hausdorff.
\end{proof}

\section{Continuous Functions}
\subsection{Continuity of a Function}

\section{Basic Examples of Topological Spaces}
\subsection{The Order Topology}
If $X$ is a set with a simple order $<$, we can define a standard topology for $X$ using the order relation, called the \emph{order topology}.
Given elements $a,b \in X$ with $a<b$, we have four intervals determined by $a,b$ in the usual ways; one open, two half-open, and one closed interval.
\begin{definition}
  Let $X$ be a simply ordered set with more than one element, and let $\SB$ be the collection of sets of the following types:
  \begin{enumerate}[label={(\arabic*)}]
    \item All open intervals $(a,b) \subset X$.
    \item All intervals of the form $[a_0,b)$, where $a_0$ is the smallest element (if any) of $X$.
    \item All intervals of the form $(a,b_0]$, where $b_0$ is the largest element (if any) of $X$.
  \end{enumerate}
  The collection $\SB$ is a basis for a topology on $X$, which is called the \emph{order topology}.
\end{definition}
Note that every element lies in a basis element, satisfying the first condition.  
Second, the intersection of any two sets of these types is another set of these types if nonempty; hence this is a basis.

Some examples are useful: this is the standard topology for $\RR$.
This acts as expected on $\RR^2$ with dictionary order.
This is the discrete topology on $\ZZ_+$;
however, this is not the discrete topology on $\cbr{1,2} \times \ZZ_+$ in dictionary order, since any basis element about $2 \times 1$ must contain some $1 \times n$, so $\{2 \times 1\}$ is not open.

We may also define \emph{rays} $(a,+\infty) = \cbr{x \in X \mid x > a}$, as well as defining the other open and closed rays analogously.
Note that the open rays (non-inclusive) are open, since $(a,+\infty)$ either equals $(a,b_0]$ in the case of a largest element, or the union of all $(a,x)$ for $x > a$ otherwise (similar proof for the open ray to $-\infty$).
In fact, the open rays are a subbasis for the order topology on $X$, since $(a,b) = (-\infty,b) \cap (a,\infty)$, and the half-open basis elements are open rays if they exist.

\subsection{The Product Topology on \texorpdfstring{$X \times Y$}{XxY}}
\begin{definition}
  Let $X$ and $Y$ be topological spaces.
  The \emph{product topology} on $X \times Y$ is the topology having as basis the collection $\SB$ of all sets of the form $U \times V$, where $U \subset X$ and $V \subset Y$ are open.
\end{definition}

Let's check that $\SB$ is a basis.
The condition clearly all $x \times y$ are in a basis element, since $X \times Y$ is a basis element.
Furthermore, if $U_i \subset X$, $V_i \subset Y$ open, 
\[
  (U_1 \times V_1) \cap (U_2 \times V_2)
  = (U_1 \cap U_2) \times (V_1 \cap V_2).
\]
Each side of this is open, so this is a basis element, and $\SB$ is a basis.

\begin{theorem}
  If $\SB$ is a basis for the topology of $X$ and $\SC$ is a basis for the topology of $Y$, then the collection
  \[
    \SD = \cbr{B \times C \mid B \in \SB \text{ and } C \in \SC}
  \]
  is a basis for the topology of $X \times Y$.
\end{theorem}
\begin{proof}
  Let $U \times V$ be an open subset of $X \times Y$ containing $x \times y$.
  Let $B,C$ be basis elements with $x \in B \subset U$ and $y \in C \subset V$ by an earlier lemma; then $x \times y \in B \times C \subset U \times V$, and $B \times C \in \SD$, so $\SD$ is a basis.
\end{proof}

Note: we will call the \emph{standard} topology the product of two copies of the standard topology on $\RR$.
We'll seek to find a subbasis for the product topology.
\begin{definition}
  Let $\pi_1:X \times Y \rightarrow X$ be defined by $(x,y) \mapsto x$, and $\pi_2:X \times Y \rightarrow Y$ be defined by $(x,y) \mapsto y$.
  The maps $\pi_i$ are called \emph{projections} of $X \times Y$ onto it's $i$th factors.
\end{definition}

The word onto was used because $\pi_i$ are surjective assuming $X$ and $Y$ are nonempty (otherwise $X \times Y$ is empty).
If $U$ is an open subset of $X$, then $\pi_1^{-1}(U) = U \times Y$ is open in $X \times Y$ (similarly for preimages in the second coordinate).

\begin{theorem}
  The collection
  \[
    \CS = \cbr{ \pi_1^{-1}(U) \mid U \text{ open in } X } \cup \cbr{ \pi_2^{-1}(V) \mid V \text{ open in } Y}
  \]
  is a subbasis for the product topology on $X \times Y$.
\end{theorem}
\begin{proof}
  Let $\ST$ be the product topology on $X \times Y$, and let $\ST'$ be the topology generated by $\CS$ as a subbasis.
  Every element of $\CS$ belongs to $\ST$, so finite intersections do as well, and arbitrary unions of finite intersections; $\ST' \subset \ST$.
  Furthermore, for any $U \times V \in \ST$, $\pi_1^{-1}(U) \cap \pi_2^{-1}V = U \times V$, so $\ST \subset \ST'$.
\end{proof}

\subsection{The Subspace Topology}
\begin{definition}
  Let $X$ be a topological space with topology $\ST$.
  If $Y  \subset X$, the collection
  \[ \ST_Y = \cbr{Y \cap U \mid U \in \ST} \]
  is a topology on $Y$, called the \emph{subspace topology}.
  With this topology, $Y$ is a \emph{subspace} of $X$.
\end{definition}

\begin{lemma}
  If $\SB$ is a basis for the topology of $X$ then the collection
  \[
    \SB_Y = \cbr{B \cap Y \mid B \in \SB}
  \]
  is a basis for the subspace topology on $Y$
\end{lemma}
\begin{proof}
  Let $U \cap Y$ be open containing $x \times y$, and let $B \in \SB$ satisfy $x \in \SB \subset U$.
  Then $x \times y \in B \cap Y \subset U \cap Y$, so $\SB_Y$ is a basis for the subspace topology.
\end{proof}

\begin{lemma}
  Let $Y$ be a subspace of $X$. If $U$ is open in $Y$ and $Y$ is open in $X$, then $U$ is open in $X$.
\end{lemma}

We'll explore the relation of the subspace topology to the order and product topologies;
this is well behaved for the product topology, but not the order topology.

\begin{theorem}
  If $A$ is a subspace of $X$ and $B$ is a subspace of $Y$, then the product topology on $A \times B$ is the same as the topology $A \times B$ inherits as a subspace of $X \times Y$.
\end{theorem}
\begin{proof}
  The general basis element for the subspace topology on $A \times B$ is $(U \times V) \cap (A \times B)$.
  Since
  \[
    (U \times V) \cap (A \times B) = (U \cap A) \times (V \cap B),  
  \]
  the bases for the subspace and product topologies are the same, and hence taking subspace and product topologies commutes.
\end{proof}

Note that this property with commuting doesn't hold for the order topology. 
An example is $Y = [0,1) \cap \{2\} \subset \RR$. 
In the subspace topology, $\{2\}$ is open, but in the order topology, any basis element containing $2$ must also contain some other elements of $Y$, so $\{2\}$ is not open with respect to the order topology.

Given an ordered set $X$, we'll say that a subset $Y$ of $X$ is \emph{convex} in $X$ if for each pair of points $a<b$ in $Y$, the interval $(a,b)$ of points in $X$ lies in $Y$; this includes intervals and rays in any $X$.

\begin{theorem}
  Let $X$ be an ordered set in the order topology; let $Y$ be a convex subset of $X$.
  Then the order topology on $Y$ is the same as the topology $Y$ inherits as a subspace of $X$.
\end{theorem}
\begin{proof}
  I'll go with a sketch, since the proof is a bit tedious.
  An open ray in $X$ intersected with $Y$ is an open ray in $Y$, and any open ray of $Y$ is the intersection of open rays in $X$.
  Since the open rays are a subbasis of the order topology, one can pull each direction of an inclusion from this.
\end{proof}

For clarity, we always assume subsets are given the subspace topology unless specified otherwise.

\end{document}

\documentclass[10pt]{report}
\usepackage{../style}

\begin{document}
\section{Topological Spaces and Bases (\S M12-13)}
\subsection{Topological Spaces}
\begin{definition}
  A \emph{topology} on a set $X$ is a collection $\ST$ of subsets of $X$ having the following properties:
  \begin{enumerate}[label={(\arabic*)}]
    \item $\varnothing$ and $X$ are in $\ST$.
    \item The union of any subcollection of $\ST$ is in $\ST$.
    \item The intersection of any finite subcollection of $\ST$ is in $\ST$.
  \end{enumerate}
  A set $X$ for which a topology $\ST$ has been specified is a \emph{topological space}, and any subset $U \subset X$ is open if $U \in \ST$.
\end{definition}

  Examples of this include the \emph{discrete topology}, where every subset is open, and the \emph{indiscrete} or \emph{trivial topology}, where only $\varnothing$ and $X$ are open.
  
  Another important (probably) example is the \emph{finite complement topology}:
  this is $(X,\ST_f)$, where $\ST_f$ is the collection of subsets $U \subset X$ such that $X - U$ is either finite or all of $X$.
  It's easy to check that $X$, $\emptyset$ work, so we note the following facts for an indexed family of nonempty elements $\{U_\alpha\} \subset \ST_f$:
  \[ X - \bigcup U_\alpha = \bigcap (X - U\alpha;) \]
  \[ X - \bigcap_{i=1}^n U_i = \bigcup_{i = 1}^n (X - U_i). \]
  From this, one can see that arbitrary unions and finite intersections work. 

\begin{definition}
  Suppose that $\ST$, $\ST'$ are topologies on $X$.
  If $\ST' \supset \ST$, then $\ST'$ is \emph{finer} than $\ST$ (and \emph{strictly finer} with a proper inclusion).
  The reverse inclusion implies $\ST'$ is \emph{coarser} than $\ST$.
  If either of these inclusions is true, then $\ST'$ is \emph{comparable} to $\ST$.
\end{definition}

Sometimes fully specifying the elements of a topology is difficult or annoying--like in algebra, one usually specifies a topology by a subcollection of open sets which generate the whole thing.

\subsection{Bases}
\begin{definition}
  If $X$ is a set, a \emph{basis} for a topology on $X$ is a collection $\SB$ of subsets of $X$ (called \emph{basis elements}) such that
  \begin{enumerate}[label={(\arabic*)}]
    \item For each $x \in X$, there exists some $B \in \SB$ with $x \in B$.
    \item If $x \in B_1 \cap B_2$, then there exists $B_3 \in \SB$ with $x \in B_3 \subset B_1 \cap B_2$.
  \end{enumerate}
  
  If $\SB$ satisfies these conditions, then we define the \emph{topology $\ST$ generated by $\SB$} by the following:
  $U \subset X$ is open if, for each $x \in U$, there is some $B \in \SB$ with $x \in B \subset U$.
\end{definition}

  We will show that this is a topology.
  Note that $\varnothing$ satisfies the condition for openness vacuously, and all $x \in X$ are contained in a basis element by hypothesis, so $X$ is open as well.
  All that's left is that unions and finite intersections work.

  Suppose $\{U_\alpha\} \subset \ST$ is a collection of open sets.
  Then, for any element $x \in \bigcap U_\alpha$, $x \in U_\alpha$ for some $\alpha$; choose $x \in \SB$ with $x \in B \subset U_\alpha$ and note that this is also in the union.
  For the intersection, we will prove the intersection of two and use induction; suppose $U,V \in \ST$.
  Then, for any $x \in U \cap V$, choose $B_1,B_2 \in \ST$ containing $x$ so that $B_1 \subset U$ and $B_2 \subset V$; then, by definition, there exists a $B_3 \subset B_1 \cap B_2$ with $x \in B_3 \subset \ST$, giving openness.
  Induction is straightforward.

  \begin{lemma}
    Let $X$ be a set and let $\SB$ be a basis for a topology $\ST$ on $X$.
    Then $\ST$ is the collection of all unions of elements of $\SB$.
  \end{lemma}
  \begin{proof}
    That $\ST$ includes all unions is straightforward from the fact that $\ST$ is a topology.
    To show that any $U \in \ST$ is a union of elements, choose for each $x \in U$ some $B_x \in \SB$ such that $x \in B_x \subset U$.
    Then $U = \bigcup B_x$, so $U$ is a union of elements of $\SB$.
  \end{proof}

  There is one significant, potentially counter-intuitive fact to note about bases:
  the representation of any open subset as a union of basis elements is not unique.
  Topological bases seem similar to the algebraic notion of a spanning set or generating set.

  \begin{lemma}
    Let $X$ be a topological space.
    Suppose that $\SC$ is a collection of open sets of $X$ such that, for each $X$ of $X$ and each $x \in U$, there is an element $C \in \SC$ such that $x \in C \subset U$.
    Then $\SC$ is a basis for the topology of $X$.
  \end{lemma}
  \begin{proof}
    First we check that $\SC$ is a basis.
    The first condition is easy; since $X$ is open, there is an element $C \in \SC$ with $x \in C \subset X$.
    To show the intersection property, take any $x \in C_1 \cap C_2$, where $C_i \in \SC$.
    Then, since $C_1 \cap C_2$ is open, there exists an element $C_3 \in \SC$ with $x \in C_3 \subset C_1 \cap C_2$.

    Suppose $\ST$ is the set of open sets on $X$ and $\ST'$ is generated by $\SC$.
    Note that, for any $U \subset \ST$ with $x \in U$, there exists an element $C \in \SC$ with $x \in C \subset U$, so $\ST'$ is finer than $\ST$.
    Conversely, if $W \in \ST'$, then $W$ is a union of elements of $\SC$, which are each open, so $W$ is open and $\ST$ is finer than $\ST'$, giving $\ST = \ST'$.
  \end{proof}

  \begin{lemma}
    Let $\SB$ and $\SB'$ be bases for the topologies $\ST$ and $\ST'$ on $X$.
    Then the following are equivalent:
    \begin{enumerate}[label={(\arabic*)}]
      \item $\ST'$ is finer than $\ST$
      \item For each $x \in X$ and each basis element $B \in \SB$ containing $x$, there is a basis element $B' \in \SB'$ such that $x \in B' \subset B$.
    \end{enumerate}
  \end{lemma}
  \begin{proof}
    $(2) \implies (1)$. Suppose $U \in \ST$.
    Then for each $x \in U$, there is a $B_x \in \ST$ containing $x$ as a subset of $U$.
    By hypothesis, there exists a $B_x' \in \ST'$ with $x \in B_x' \subset B_x \subset U$, so $U$ is the union of elements of $\SB'$ and $U \in \ST'$.

    $(1) \implies (2)$. Suppose $U \in \ST$.
    Then, for each $x \in U$ and $B \in \SB$ containing $x$, by condition $(1)$ and the definition of generation, $B \in \ST'$; since $\ST'$ is generated by $\SB'$, there is an element $B' \in \SB'$ such that $x \in B' \subset B$.
  \end{proof}
  
  We will move on to our first serious examples of topological spaces:
  \begin{definition}
    If we denote the collection of all open intervals on the real line $\SB$, 
    \[ \prn{a,b} = \cbr{x \mid a < x < b }, \]
    the topology generated by $\SB$ is the \emph{standard topology} on the real line.
    This is the topology assumed for $\RR$ unless otherwise specified.

    If $\SB'$ is the half open intervals, we will call the topology generated by $\SB'$ the \emph{lower limit topology} on $\RR$; denote $\RR$ with this topology by $\RR_l$.

    If we let $K$ be the set of numbers of the form $1/n$ and let $\SB''$ be the collection of all open intervals $(a,b)$ and sets of the form $(a,b) - K$, we call the topology generated by $\SB''$ the \emph{K-topology} on $\RR$, denoted by $\RR_k$.
  \end{definition}

  \begin{lemma}
    The topologies of $\RR_l$ and $\RR_k$ are strictly finer than the standard topology on $\RR$, but are not comparable with one another.
  \end{lemma}
  \begin{proof}
    Let $\ST$, $\ST_l$, and $\ST_k$ be the topologies on $\RR$, $\RR_l$, and $\RR_k$.
    To show that $\ST_l$ is finer than $\ST$, take any $(a,b) \in \ST$ and $x \in (a,b)$; then $[x,b)$ contains $x$ and lies in $(a,b)$.
    However, no $(c,d) \subset [x,b)$ contains $x$, so $\ST_l \supsetneq \ST$.
    To show that $\ST_k$ is finer than $\ST$, any $(a,b) \in \ST$ has $(a,b) \in \ST_k$ by construction.
    However, $(-1,1) - K$ has no subset in $\ST$ containing 0, giving $\ST_k \supsetneq \ST$.
  
    To see that these are not comparable, use a similar argumen to before.
    There is no $(c,d) - K \subset [x,b)$ containing $x$, and there is no $[a,b) \subset (-1,1)-k$ contining $0$.
  \end{proof}

  One last basic\footnote{This was intended to be a really bad pun} definition:
  \begin{definition}
    A subbasis $\CS$ for a topology on $X$ is a collection of subsets of $X$ whose union equals $X$.
    The \emph{topology generated by the subbasis $\CS$} is defined to be the collection $\ST$ of all unions of finite intersections of the elements of $\CS$.
  \end{definition}

  We'll check that $\ST$ is a topology.
  By an earlier lemma, it suffices to show that the collection of finite intersections of $\CS$ is a basis.
  Given $x \in X$, it belongs to an element of $\CS$ and hence an element of $\SB$, giving the first condition.
  To show the second condition, let $B_1 = S_1 \cap \dots \cap S_m$ and $B_2 = S_1' \cap \dots \cap S_n'$ be any two elements of $\SB$; then $B_1 \cap B_2 = S_1 \cap \dots \cap S_m \cap S_1' \cap \dots \cap S_n' \in \SB$, making $\SB$ a basis.

  \section{Basic Examples of Topological Spaces}
\end{document}

\documentclass[10pt]{report}
\usepackage{../style}

\begin{document}
\section{Connected Spaces}
\begin{definition}
  Let $X$ be a topological space.
  A \emph{separation} of $X$ is a pair $U,V$ of disjoint nonempty open subsets of $X$ whose union is $X$.
  The space $X$ is said to be \emph{connected} if there does not exist a separation of $X$.
\end{definition}

Connectedness is a topological property, so any space homeomorphic to a connected space is also connected.
There is an equivalent definition:
A space $X$ is connected iff the only subsets of $X$ that are both open and closed in $X$ are the empty set and $X$ itself.

\begin{lemma}
  If $Y$ is a subspace of $X$, a separation of $Y$ is a pair of disjoint nonempty sets $A$ and $B$ whose union is $Y$, neither of which contains a limit point of the other.
  The space $Y$ is connected if there exists no separation of $Y$.
\end{lemma}
\begin{proof}
  Suppose $A$ and $B$ form a separation of $Y$.
  Then $A$ is clopen.
  The closure of $A$ in $Y$ is $\bar A \cap Y$, and $A$ is closed, so $\bar A \cap Y = A$; then $B$ contains no limit points of $A$.
  The same argument gives that $A$ contains no limit points of $B$.

  Suppose that $A$ and $B$ satisfy the disjointness condition.
  Then $\bar A \cap B = \varnothing$, meaning$\bar{A} \cap Y = A$, so $A$ is closed in $Y$; the same argument gives that $B$ is closed in $Y$.
  Hence $A$ and $B$ are both also open in $Y$, and they form a separation of $Y$.
\end{proof}

\begin{lemma}
  If the sets $C$ and $D$ form a separation of $X$, and if $Y$ is a connected subspace of $X$, then $Y$ lies entirely within either $C$ or $D$.
\end{lemma}
\begin{proof}
  Suppose $Y$ is a subspace of $X$ that intersects $C$ and $D$.
  Then $Y \cap C$ and $Y \cap D$ are both nonempty, closed, and open with respect to $Y$, so $Y$ is separated.
\end{proof}

We'll use these lemmas to find a couple of ways to find new connected spaces from spaces already known to be connected.

\begin{theorem}
  The union of a collection of connected subspaces of $X$ that have a point in common is connected.
\end{theorem}
\begin{proof}
  Let $\cbr{A_\alpha}$ be a collection of connected subspaces of $X$ that share the point $p$, and let $Y = \bigcup A_\alpha$.
  Suppose for contradiction that the sets $C$ and $D$ are a separation of $Y$.
  $p$ must lie within $C$ or $D$; suppose $p \in C$.
  Then $A_\alpha \subset C$ for all $\alpha$ by the lemma, so $D$ is empty, a contradiction.
\end{proof}

\begin{theorem}
  Let $A$ be a connected subspace of $X$.
  If $A \subset B \subset \bar A$, then $B$ is also connected.
\end{theorem}
\begin{proof}
  Suppose $C \cup D$ is a separation of $B$; then $A$ must lie in either $C$ or $D$.
  Suppose $A \subset C$.
  Then $D$ contains no limit points of $A$ or points of $A$, so $D$ is empty, leading to contradiction.
\end{proof}

\begin{theorem}
  The image of a connected space under a continuous map is connected.
\end{theorem}
\begin{proof}
  Let $f:X \rightarrow Y$ be continuous and let $B \cup C$ be a separation of $\text{im}\, f$.
  Then $f^{-1}(B) \cup f^{-1}(C)$ is a separation of $X$ (open by continuity, disjoint by function, nonempty by surjectivity), so $X$ is not connected.
\end{proof}

\begin{theorem}
  A finite cartesian product of connected spaces is connected.
\end{theorem}
\begin{proof}
  Let $X$ and $Y$ be connected spaces with $b \in Y$, $x \in X$.
  Note that $X \times b$ and $x \times Y$ are the image of connected spaces, so they are connected, and they share a point, so $T_x = (X \times b) \cup (x \times Y)$ is connected.
  Then, $\bigcup_{x \in X} T_x$ is connected and equals the cartesian product.
\end{proof}


\section{Connected Subspaces of the Real Line}
Here we'll deal with the real line, and prove a generalized IVT as well as connectivity of intervals, rays, etc.
We'll start with a generalization:
\begin{definition}
  A simply ordered set $L$ having more than one element is called a \emph{linear continuum} if the following hold:
  \begin{enumerate}[label={(\arabic*)}]
    \item $L$ has the least upper bound property.
    \item If $x<y$ there exists $z$ such that $x < z < y$.
  \end{enumerate}
\end{definition}

\begin{theorem}
  If $L$ is a linear continuum in the order topology, then $L$ is connected, and so are intervals and rays in $L$.
\end{theorem}
\begin{proof}
  We will prove this of a convex subset $Y$ of $L$, which concerns all of the given cases.
  Suppose that $A \cup B$ is a separation of $Y$, and choose $a \in A$ and $b \in B$.
  For convenience suppose that $a<b$.
  We know $\brk{a,b}$ is contained in $Y$, so $[a,b]$ is the union of the disjoint sets
  \[
    A_0 = A \cap \brk{a,b} \text{\hspace{10pt} and \hspace{10pt}} B_0 = B \cap \brk{a,b},
  \]
  each of which is open in $\brk{a,b}$ in the subspace topology, which is equivalent to the order topology for convex sets.\footnote{Think about open rays.}
  Let $c = \sup A_0$; we will show that $c$ isn't in $A_0$ or $B_0$, a contradiction.

  \emph{Case 1.} Suppose that $c \in B_0$.
  Since $B_0$ is open in $\brk{a,b}$, there is some interval $(d,c]$ contained in $B_0$.
  Since $c \neq a$, $a < c < b$ or $c = b$. 
  If $a<c<b$, we know that $(c,b]$ doesn't intersect $A_0$, so $(d,b]$ is contained in $B_0$.
  Hence $d$ is a smaller upper bound on $A_0$ then $c$, a contradiction.
  
  \emph{Case 2.} Suppose $c \in A_0$.
  Then be openness, there exists some $[c,e) \subset A_0$, which must contain some $c < z < e$, contradicting $c$ as an upper bound.
\end{proof}

\begin{corollary}
  The real line $\RR$ is connected, and so are intervals and rays in $\RR$.
\end{corollary}

\begin{theorem}
  Let $f:X \rightarrow Y$ be a continuous map, where $X$ is a connected space and $Y$ is an ordered set in the order topology.
  If $a$ and $b$ are two points of $X$ and if $r$ is a point of $Y$ lying between $f(a)$ and $f(b)$, then there exists a point $c$ of $X$ such that $f(c) = r$.
\end{theorem}
\begin{proof}
  Assume the hypothesis, and consider the disjoint nonempty open sets $A = f(X) \cap \prn{-\infty,r}$ and $B = f(X) \cap \prn{r,\infty}$.
  If there was no point $c$ of $X$ such that $f(c)=r$, this would constitute a separation of $f(X)$; however, $X$ is connected, so $f(X)$ must be connected.
\end{proof}

We'll now move on to path connectedness.

\begin{definition}
  Given points $x$ and $y$ of the space $X$, a \emph{path} in $X$ from $x$ to $y$ is a continuous map $f:[a,b] \rightarrow X$ from a closed interval on the real line such that $f(a) = x$ and $f(b) = y$.
  A space $X$ is \emph{path connected} if every pair of points of $X$ can be joined by a path in $X$.
\end{definition}

It's easy to see that path connectedness implies connectedness.
Suppose $X$ is a space with separation $A \cup B$, and let $f:[a,b] \rightarrow X$ be a path in $X$.
Then $f([a,b])$ must be connected, so it lies entirely in $A$ and $B$; no path connects points of $A$ with points of $B$.

Let's explore a couple of examples: \emph{the ordered square $I_0^2$ is connected but not path connected}.

Since $I_0^2$ is a linear continuum, it's connected. 
Let $p=0 \times 0$ and $q = 1 \times 1$, and suppose there is a path $f:\brk{a,b} \rightarrow I_0^2$.
Then by IVT, each $x \in I$ has a nonempty subset $U_x = f^{-1}\prn{x \times (0,1)}$.
Each $U_x$ contains at least one unique rational number, so $\cbr{U_x}_{x \in I}$ is at most countable, contradicting the uncountability of $I$.

Another example is the \emph{topologist's sine curve}, the closure of the following subset of the plane:
\[
  S = \cbr{x \times sin(1/x) \mid 0 < x \leq 1}.
\]
$S$ is connected as the image of a connected set, so $\bar S$ is also connected.
Visually, it is the union of $S$ and the vertical interval $0 \times \brk{-1,1}$.
However, it is not path connected.

Suppose $f:\brk{a,c} \rightarrow \bar S$ is a path beginning at the origin and ending at a point of $S$.
The set of $t$ for which $f(t) \in 0 \times \brk{-1,1}$ is closed, so it has a largest element $b$.
Then $f:\brk{b,c} \rightarrow \bar S$ is a path that maps $b$ onto the vertical interval $0 \times \brk{-1,1}$ and maps the other points to $S$.

For convenience replace $[b,c]$ by $[0,1]$, and let $f(t) = \prn{x(t),y(t)}$.
Then $x(0) = 0$, and when $t>0$ $x(t) > 0$ and $y(t) = sin(1/x(t))$.
We'll find a sequence of points $t_n \rightarrow 0$ such that $y(t_n) = (-1)^n$, giving nonconvergence and contradicting continuity.

Given $n$, choose $u$ with $0 < u < x(1/n)$ such that $sin(1/u) = (-1)^n$.
Then use IVT to find $t_n$ with $0 < t_n < 1/n$ such that $x(t_n) = u$.

\end{document}

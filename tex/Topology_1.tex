\documentclass{report}
\usepackage{../style}
\begin{document}
\section{Relations (\S M3) }
Here we'll summarise a chapter of mostly-redundant logic and set theory.
\begin{definition}
  A \emph{relation} on a set $A$ is a subset $C$ of the cartesian product $A \times A$.
\end{definition}
We'll use the notation $xCy$ to mean $(x,y) \in C$, verbalizing this  as ``$x$ is in the relation $C$ to $y$''.
\begin{definition}
  A relation $C$ on a set $A$  is an \emph{equivalence relation} if it is reflexive, symmetric, and transitive relation.
  Equivalence relations and partitions are precisely the same thing.
  The generic symbol for equivalence relations is $\sim$.
\end{definition}
\begin{definition}
  A relation $C$ on a set $A$ is an \emph{order relation} (or a \emph{simple} or \emph{linear order}) if it has the following properties:
  \begin{enumerate}[label={(\arabic*)}]
    \item (Comparability) For every $x,y \in A$ for which $x \neq y$, either $xCy$ or $yCx$.
    \item (Nonreflexivity) For no $x$ in $A$ does the relation $xCx$ hold.
    \item (Transitivity) If $xCy$ and $yCz$, then $xCz$.
  \end{enumerate}
  The generic symbol for an order relation is $<$.
\end{definition}

We can move on to less familiar, but similarly simple matters of notation:
\begin{definition}
  If $X$ is a set and $<$ is an order relation on $X$, and if $a < b$, we use the notation $\prn{a,b}$ to denote the set 
  \[
    \{ x \in X \mid a < x < b \};
  \]
  and we call this an \emph{open interval} in $X$. 
  If this set is empty, we call $a$ the \emph{immediate predecessor} of $b$ and $b$ the \emph{immediate successor} of $a$.
\end{definition}

\begin{definition}
  Suppose that $A$ and $B$ are two sets with order relations $<_A$ and $<_B$.
  We say that $A$ and $B$ have the same \emph{order type} if there is a bijective correspondence that preserves order;
  i.e. there is a bijection $f:A \rightarrow B$ with $a <_A b$ iff $f(a) <_B f(b)$.
\end{definition}

\begin{definition}
  Suppose that $A$ and $B$ are two sets with order relations $<_A$ and $<_B$.
  We define the \emph{dictionary order relation} $<$ on $A \times B$ by defining
  \[
    a_1 \times b_1 < a_2 \times b_2 \text{ if } a_1 <_A a_2 \text{ or if } a_1 = a_2 \text{ and } b_1 <_B b_2
  \]
\end{definition}

\section{Integers and Real Numbers (\S M4)}
\begin{definition}
  A subset $A$ of the real numbers is said to be \emph{inductive} if it contains the number 1, and if for every $x \in A$, $x + 1 \in A$.
  Let $\SA$ be the collection of all inductive subsets of $\RR$. Then the set $\ZZ_+$ of \emph{positive integers} is defined by the equation
  \[
    \ZZ_+ = \bigcap_{A \in \SA} A.
  \]
  Note that $\ZZ_+$ is inductive, and it has no inductive proper subsets.
  The integers and rationals can be defined from this.
  We define a \emph{section} of the positive integers $\{1,\dots,n\} = S_{n+1}$.
\end{definition}
We'll move on to prove some nontrivial properties of the integers:
\begin{theorem}
  {\normalfont (Well-ordering property)} Every nonempty subset of $\ZZ_+$ has a smallest element.
\end{theorem}
\begin{proof}
  We'll first prove this for any nonempty subset $S_n \subset \ZZ_+$.
  Let $A$ be the set of positive integers such that this holds.
  Clearly $1 \in A$ so suppose $n \in A$ and $\emptyset \neq C \subset S_{n+1}$.
  Then either $C = \{n\}$ or $C \cap S_n$ contains the smallest element;
  hence $A = \ZZ_+$ by induction.

  Suppose $D$ is a nonempty subset of $\ZZ_+$ and choose an element $n$ of $D$; then $A = D \cap S_{n+1}$ is nonempty and has a minimal element, which is also the  minimal element of $D$.
\end{proof}

\begin{theorem}
  {\normalfont (Strong induction principle)} Let $A$ be a set of positive integers. Suppose that, for each integer $n$, $S_n$ implies $n \in A$. Then $A = \ZZ_+$.
\end{theorem}
\begin{proof}  
If $A \neq \ZZ_+$, let $n$ be the smallest positive integer not in $A$; then $S_n \subset A$, leading to contradiction.
\end{proof}

\section{Cartesian Products (\S M5)}
\begin{definition}
  Let $m$ be a positive integer.
  Given a set $X$, we define an \emph{m-tuple} of elements of $X$ to be a function
  \[
    x : \{1,\dots,m\} \rightarrow X.
  \]
  We sometimes write this as $(x_1,\dots,x_m)$ where $x_m$ is the $m$th coordinate of $x$.
  We define the \emph{cartesian product} of an indexed family
  \[
    \prod_{i=1}^m A_i \text { \hspace{20pt} or \hspace{20pt} } A_1 \times \dots \times A_m,
  \]
  to be the set of all $m$-tuples of elements of $X$ such that $x_i \in A_i$ for each $i$.
\end{definition}
\begin{definition}
  Given a set $X$, we define an \emph{$\omega$-tuple} of elements of $X$ to be a function
  \[
     x : \ZZ_+ \rightarrow X;
  \]
  this is also called a \emph{sequence}.
  We define a cartesian product similarly to with finite tuples, and write $X^\omega = X \times X \times \dots$.
\end{definition}

\section{Infinite Sets and the Axiom of Choice (\S M9)}
This section will lead into discussion about the Axiom of Choice.
\begin{theorem}
 Let $A$ be a set. The following statements about $A$ are equivalent:
 
 \begin{enumerate}[label={(\arabic*)}]
    \item There exists an injective function $f:\ZZ_+ \rightarrow A$.
    \item There exists a bijection of $A$ with a proper subset of itself.
    \item $A$ is infinite.
 \end{enumerate}
\end{theorem}
\begin{proof}
  Suppose $f: \ZZ_+ \rightarrow A$ injective.
  Denote $f(n)$ be $a_n$ and $f(\ZZ_+)$ by $B$.
  Then we can define $g:A \rightarrow A - \{a_1\}$ by $g(x) = x$ for $x \in A - B$ and $g(a_n) = a{n+1}$; this gives $1 \implies 2$, and $2 \implies 3$ by contraposition.

  Suppose $A$.
  Then choose some $a_1$, and if possible, choose an $a_n$ such that $a_n \in A - \{a_1,\dots,a_{n-1}\}$.
  This must always be possible since $A$ is not finite; hence $3 \implies 1$.
\end{proof}

Note something implicit and dangerous, this defined a subset $\ZZ_+ \times A$, which is not possible within ZF. We need another way of asserting the existence of this set:

\begin{axiom}
  Given a collection $\SA$ of disjoint nonempty set, there exists a set $C$ consisting of exactly one element from each element of $\SA$; 
  that is, a set $C$ such that $C$ is contained in the union of the elements of $\SA$, and for each $A \in \SA$, the set $C \cap A$ contains a single element.
\end{axiom}

\begin{lemma}
  {\normalfont (Existence of a choice function.)}
  Given a collection $\SB$ of nonempty sets (not necessarily disjoint), there exists a \textbf{choice function}
  \[
    c: \SB \rightarrow \bigcup_{B \in \SB} B
  \]
  such that $c \prn{B}$ is an element of $B$, for each $B \in \SB$.\footnote{This lemma does not assume disjoint sets, unlike AOC.}
\end{lemma}
\begin{proof}
  Given an element $B$ of $\SB$, define a set $B'$ as follows:
  \[
    B' = \cbr{ \prn{B,x} \mid x \in B} \subset \SB \times \bigcup_{B \in \SB} B.
  \]
  Because $B$ contains at least one element $x$, $B'$ contains at least $\prn{B,x}$, so it is nonempty.

  We claim that, if $B_1,B_2 \in \SB$ distinct, then the corresponding sets $B_1'$ and $B_2$' are distinct, because any elements of $B_1'$ and $B_2'$ are $(B_1,x_1)$ and $(B_2,x_2)$, which can not be equal because they have distinct first coordinates. 
  Now, form the collection
  \[
    \SC = \cbr{B' \mid B \in \SB}.
  \]
  By the axiom of choice, there exists a set $c$ consisting of exactly one element of $\SC$.
  That is, $c$ contains exactly one element from each $B'$, so $c$ is a rule for the function from the collection $\SB$ to the set $\bigcup_{B \in \SB} B$.
  If $\prn{B,x} \in c$, $x \in B$, so $c(B) \in B$ as desired.
\end{proof}

We can use a choice function to make the proof for the beginning theorem more precise; it is true only assuming choice.
However, in practice, this will usually be omitted with the general understanding that the first proof, involving an infinite number of arbitrary choices, can be made precise using the axiom of choice if needed.

The \emph{finite axiom of choice}, involving a finite collection of disjoint nonempty sets, is uncontested.
Munkres  didn't mention this, but know about countable choice as well.

\section{Well-Ordered Sets (\S M10)}
\begin{definition}
  A set $A$ with an order relation $<$ is said to be \emph{well-ordered} if every nonempty subset of $A$ has a smallest element.
\end{definition}

\begin{theorem}
  Every nonempty finite ordered set has the order type of a section $\cbr{1,\dots,n}$ of $\ZZ_+$, so it is well-ordered.\footnote{This is trivial}
\end{theorem}
 
\begin{theorem}
  {\normalfont (Well-ordering theorem)} If $A$ is a set, there exists an order relation on $A$ that is a well-ordering.\footnote{This assumes AOC and is long but not exceedingly difficult}
\end{theorem}

\begin{definition}
  Let $X$ be a well-ordered set. Given $\alpha \in X$, let the \emph{section} of $X$ by $\alpha$ $S_\alpha$ denote the set
  \[
    S_\alpha = \cbr{ x \mid x \in X \text{ and } x < \alpha}.
  \]
\end{definition}

\begin{lemma}
  There exists a well-ordered set $A$ having a largest element $\Omega$, such that the section $S_\Omega$ of $A$ by $\Omega$ is uncountable but every other section of $A$ is countable.
\end{lemma}
\begin{proof}
We begin with an uncountable well-ordered set $B$.
Let  $C$ be the well-ordered set $\cbr{1,2} \times B$ in dictionary order; then some section of $C$ is uncountable.
Let $\Omega$ be the smallest element of $C$ for which the section of $C$ by $\Omega$ is uncountable. Then let $A$ consist of this section consist of this section along with the element $\Omega$.
\end{proof}

Note that $S_\Omega$ has an order type uniquely determined by this condition, making it a \emph{minimal uncountable well-ordered set}. 
Furthermore, we'll denote $A = S_\Omega \cup \cbr{\omega}$ as $\overline S_\Omega$.
We'll see why this is useful in the next theorem:
\begin{theorem}
  If $A$ is a countable subset of $S_\Omega$, then $A$ has an upper bound in $S_\Omega$.
\end{theorem}
\begin{proof}
  Let $A$ be a countable subset of $S_\Omega$.
  For each $a \in A$, the section $S_a$ is countable;
  hence $B = \cup_{a \in A} S_a$ is also countable.
  Since $S_\omega$ is uncountable, there exists an $x \in S_\Omega - B$;
  then $x$ is an upper bound for $A$.
\end{proof}

\section{The Max imum Principle (\S M11)}
\begin{definition}
  Given a set $A$, a relation $\prec$ on $A$ is called a \emph{strict partial order} on $A$ if it has the following two properties:
  \begin{enumerate}[label={(\arabic*)}]
    \item (Nonreflexivity) The relation $a \prec a$ never holds.
    \item (Transitivity) If $a \prec b$ and $b \prec c$, then $a \prec c$.
  \end{enumerate}
  That is, a strict partial order is a simple order without the requirement of comparability.
\end{definition}

This is useful for the Maximum principle, or Zorn's lemma.
\begin{theorem}
  {\normalfont (Maximum principle)}
  Let $A$ be a set; let  $\prec$ be a strict partial order on $A$.
  Then there exists a maximal simply ordered subset $B$ of $A$.\footnote{Munkres didn't prove this, so I won't. Intuition: index using well-ordering theorem, put items comparable to what is ``in a box'' in that box etc}
\end{theorem}

\begin{definition}
  Let $A$ be a set and let $\prec$ be a strict partial order on $A$.
  If $B$ is a subset of $A$, an \emph{upper bound} on $B$ is an element $c \in A$ such that, for every $b \in B$, $b \preceq c$.
  A \emph{maximal element} of $A$ is an element $m$ of $A$ such that $m \not\prec a$ for all $a \in A$.
\end{definition}

\begin{lemma}
  {\normalfont (Zorn's Lemma)}
  Let $A$ be a set that is strictly partially ordered. 
  If every simply ordered subset of $A$ has an upper bound in $A$, has an upper bound in $A$ tjen $A$ has a maximum element.
\end{lemma}

Zorn's lemma, the maximum principle, the well-ordering theorem, and the axiom of choice are all equivalent.
Note: some authors formulate these things on a partial ordering $\preceq$, depending on taste.
\end{document}
